%
% The Lion's Commentary, file ch23.tex, version 1.4, 16 May 1994
%
\se{Character Handling}

Buffering for character special devices
is provided via a set of four word
blocks, each of which provides storage
for six characters. The prototype
storage block is ``cblock'' (8140) which
incorporates a word pointer (to a similar
structure) along with the six characters.

Structures of type ``clist'' (7908) which
contain a character counter plus a head
and tail pointer are used as ``headers''
for lists of blocks of type ``cblock''.

``cblock''s which are not in current use
are linked via their head pointers into
a list whose head is the pointer
``cfreelist'' (3149). The head pointer
for the last element of the list has
the value ``NULL''.

A list of ``cblock''s provides storage
for a list of characters. The procedure
``putc'' may be used to add a character
to the tail of such a list, and ``getc'',
to remove a character from the head of
such a list.

Figures 23.1 through 23.4 illustrate
the development of a list as characters
are deleted and added.

%
% The Lion's Commentary, file fig23_1.tex, version 1.2, 15 May 1994
%
\vspace{0.6cm}
\setlength{\unitlength}{0.0109in}%
\begin{picture}(245,155)(40,525)
\thicklines
\put(185,540){\framebox(25,140){}}
\put(195,650){\makebox(0,0)[lb]{\raisebox{0pt}[0pt][0pt]{\twlrm g}}}
\put(195,629){\makebox(0,0)[lb]{\raisebox{0pt}[0pt][0pt]{\twlrm h}}}
\put(195,608){\makebox(0,0)[lb]{\raisebox{0pt}[0pt][0pt]{\twlrm i}}}
\put(195,587){\makebox(0,0)[lb]{\raisebox{0pt}[0pt][0pt]{\twlrm j}}}
\put(195,566){\makebox(0,0)[lb]{\raisebox{0pt}[0pt][0pt]{\twlrm k}}}
\put(195,545){\makebox(0,0)[lb]{\raisebox{0pt}[0pt][0pt]{\twlrm l}}}
\put(240,540){\framebox(25,140){}}
\put(250,629){\makebox(0,0)[lb]{\raisebox{0pt}[0pt][0pt]{\twlrm n}}}
\put(250,608){\makebox(0,0)[lb]{\raisebox{0pt}[0pt][0pt]{\twlrm o}}}
\put(250,587){\makebox(0,0)[lb]{\raisebox{0pt}[0pt][0pt]{\twlrm p}}}
\put(250,566){\makebox(0,0)[lb]{\raisebox{0pt}[0pt][0pt]{\twlrm q}}}
\put(250,545){\makebox(0,0)[lb]{\raisebox{0pt}[0pt][0pt]{\twlrm r}}}
\put(247,650){\makebox(0,0)[lb]{\raisebox{0pt}[0pt][0pt]{\twlrm m}}}
\put( 40,540){\framebox(60,60){}}
\put( 90,570){\vector( 1, 0){ 40}}
\put(130,540){\framebox(25,140){}}
\put(140,670){\vector( 1, 0){ 45}}
\put(195,670){\vector( 1, 0){ 45}}
\put( 90,550){\line( 0,-1){ 25}}
\put( 90,525){\line( 1, 0){195}}
\put(285,525){\line( 0, 1){ 25}}
\put(285,550){\vector(-1, 0){ 20}}
\put( 48,584){\makebox(0,0)[lb]{\raisebox{0pt}[0pt][0pt]{\twlrm 14}}}
\put( 49,564){\makebox(0,0)[lb]{\raisebox{0pt}[0pt][0pt]{\twlrm head}}}
\put( 49,545){\makebox(0,0)[lb]{\raisebox{0pt}[0pt][0pt]{\twlrm tail}}}
\put(140,566){\makebox(0,0)[lb]{\raisebox{0pt}[0pt][0pt]{\twlrm e}}}
\put(140,545){\makebox(0,0)[lb]{\raisebox{0pt}[0pt][0pt]{\twlrm f}}}
\end{picture}
\vspace{0.4cm}
\begin{center}
{\large \bf Figure 23.1}
\end{center}
\vspace{0.2cm}


%
% The Lion's Commentary, file fig23_2.tex, version 1.2, 15 May 1994
%
\vspace{0.6cm}
\setlength{\unitlength}{0.0109in}%
\begin{picture}(245,155)(40,525)
\thicklines
\put(185,540){\framebox(25,140){}}
\put(195,650){\makebox(0,0)[lb]{\raisebox{0pt}[0pt][0pt]{\twlrm g}}}
\put(195,629){\makebox(0,0)[lb]{\raisebox{0pt}[0pt][0pt]{\twlrm h}}}
\put(195,608){\makebox(0,0)[lb]{\raisebox{0pt}[0pt][0pt]{\twlrm i}}}
\put(195,587){\makebox(0,0)[lb]{\raisebox{0pt}[0pt][0pt]{\twlrm j}}}
\put(195,566){\makebox(0,0)[lb]{\raisebox{0pt}[0pt][0pt]{\twlrm k}}}
\put(195,545){\makebox(0,0)[lb]{\raisebox{0pt}[0pt][0pt]{\twlrm l}}}
\put(240,540){\framebox(25,140){}}
\put(250,629){\makebox(0,0)[lb]{\raisebox{0pt}[0pt][0pt]{\twlrm n}}}
\put(250,608){\makebox(0,0)[lb]{\raisebox{0pt}[0pt][0pt]{\twlrm o}}}
\put(250,587){\makebox(0,0)[lb]{\raisebox{0pt}[0pt][0pt]{\twlrm p}}}
\put(250,566){\makebox(0,0)[lb]{\raisebox{0pt}[0pt][0pt]{\twlrm q}}}
\put(250,545){\makebox(0,0)[lb]{\raisebox{0pt}[0pt][0pt]{\twlrm r}}}
\put(247,650){\makebox(0,0)[lb]{\raisebox{0pt}[0pt][0pt]{\twlrm m}}}
\put( 40,540){\framebox(60,60){}}
\put( 90,570){\vector( 3,-1){ 40.500}}
\put(130,540){\framebox(25,140){}}
\put(140,670){\vector( 1, 0){ 45}}
\put(195,670){\vector( 1, 0){ 45}}
\put( 90,550){\line( 0,-1){ 25}}
\put( 90,525){\line( 1, 0){195}}
\put(285,525){\line( 0, 1){ 25}}
\put(285,550){\vector(-1, 0){ 20}}
\put( 49,564){\makebox(0,0)[lb]{\raisebox{0pt}[0pt][0pt]{\twlrm head}}}
\put( 49,545){\makebox(0,0)[lb]{\raisebox{0pt}[0pt][0pt]{\twlrm tail}}}
\put(140,545){\makebox(0,0)[lb]{\raisebox{0pt}[0pt][0pt]{\twlrm f}}}
\put( 48,584){\makebox(0,0)[lb]{\raisebox{0pt}[0pt][0pt]{\twlrm 13}}}
\end{picture}
\vspace{0.4cm}
\begin{center}
{\large \bf Figure 23.2}
\end{center}
\vspace{0.2cm}


Initially the list is assumed to contain the fourteen characters
``efghijklmnopqr''. Note that the head
and tail pointers point to {\bf characters}.
If the first character, ``e'', is removed
by ``getc'', the situation portrayed in
Figure 23.1 changes to that of Figure
23.2. The character count has been
decremented and the head pointer has
been advanced by one character position.

If a further character, ``f'', is removed
from the head of the list, the
situation becomes as in Figure 23.3.
The character count has been decremented;
the first ``cblock'' no longer
contains any useful information and has
been returned to ``cfreelist''; and the
head pointer now points to the first
character in the second ``cblock''.

%
% The Lion's Commentary, file fig23_3.tex, version 1.2, 15 May 1994
%
\vspace{0.6cm}
\setlength{\unitlength}{0.0109in}%
\begin{picture}(245,155)(40,525)
\thicklines
\put(185,540){\framebox(25,140){}}
\put(195,650){\makebox(0,0)[lb]{\raisebox{0pt}[0pt][0pt]{\twlrm g}}}
\put(195,629){\makebox(0,0)[lb]{\raisebox{0pt}[0pt][0pt]{\twlrm h}}}
\put(195,608){\makebox(0,0)[lb]{\raisebox{0pt}[0pt][0pt]{\twlrm i}}}
\put(195,587){\makebox(0,0)[lb]{\raisebox{0pt}[0pt][0pt]{\twlrm j}}}
\put(195,566){\makebox(0,0)[lb]{\raisebox{0pt}[0pt][0pt]{\twlrm k}}}
\put(195,545){\makebox(0,0)[lb]{\raisebox{0pt}[0pt][0pt]{\twlrm l}}}
\put(240,540){\framebox(25,140){}}
\put(250,629){\makebox(0,0)[lb]{\raisebox{0pt}[0pt][0pt]{\twlrm n}}}
\put(250,608){\makebox(0,0)[lb]{\raisebox{0pt}[0pt][0pt]{\twlrm o}}}
\put(250,587){\makebox(0,0)[lb]{\raisebox{0pt}[0pt][0pt]{\twlrm p}}}
\put(250,566){\makebox(0,0)[lb]{\raisebox{0pt}[0pt][0pt]{\twlrm q}}}
\put(250,545){\makebox(0,0)[lb]{\raisebox{0pt}[0pt][0pt]{\twlrm r}}}
\put(247,650){\makebox(0,0)[lb]{\raisebox{0pt}[0pt][0pt]{\twlrm m}}}
\put( 40,540){\framebox(60,60){}}
\put(195,670){\vector( 1, 0){ 45}}
\put( 90,550){\line( 0,-1){ 25}}
\put( 90,525){\line( 1, 0){195}}
\put(285,525){\line( 0, 1){ 25}}
\put(285,550){\vector(-1, 0){ 20}}
\put( 90,570){\line( 1, 0){ 50}}
\put(140,570){\line( 0, 1){ 85}}
\put(140,655){\vector( 1, 0){ 45}}
\put( 49,564){\makebox(0,0)[lb]{\raisebox{0pt}[0pt][0pt]{\twlrm head}}}
\put( 49,545){\makebox(0,0)[lb]{\raisebox{0pt}[0pt][0pt]{\twlrm tail}}}
\put( 48,584){\makebox(0,0)[lb]{\raisebox{0pt}[0pt][0pt]{\twlrm 12}}}
\end{picture}
\vspace{0.4cm}
\begin{center}
{\large \bf Figure 23.3}
\end{center}
\vspace{0.2cm}


The question now poses itself: ``how is
the difference between the first and
second situations detected so that the
action taken is always appropriate?'':

The answer (if you have not already
guessed) involves looklng at the value
of the pointer address modulo 8. Since
division by eight is easily performed
on a binary computer, the reason for
the choice of six characters per
``cblock'' should now also be apparent.

The addition of a character to the list
is illustrated in the change between
Figure 23.3  and Figure 23.4.

%
% The Lion's Commentary, file fig23_4.tex, version 1.2, 15 May 1994
%
\vspace{0.6cm}
\setlength{\unitlength}{0.0109in}%
\begin{picture}(305,155)(60,525)
\thicklines
\put(185,540){\framebox(25,140){}}
\put(195,650){\makebox(0,0)[lb]{\raisebox{0pt}[0pt][0pt]{\twlrm g}}}
\put(195,629){\makebox(0,0)[lb]{\raisebox{0pt}[0pt][0pt]{\twlrm h}}}
\put(195,608){\makebox(0,0)[lb]{\raisebox{0pt}[0pt][0pt]{\twlrm i}}}
\put(195,587){\makebox(0,0)[lb]{\raisebox{0pt}[0pt][0pt]{\twlrm j}}}
\put(195,566){\makebox(0,0)[lb]{\raisebox{0pt}[0pt][0pt]{\twlrm k}}}
\put(195,545){\makebox(0,0)[lb]{\raisebox{0pt}[0pt][0pt]{\twlrm l}}}
\put(240,540){\framebox(25,140){}}
\put(250,629){\makebox(0,0)[lb]{\raisebox{0pt}[0pt][0pt]{\twlrm n}}}
\put(250,608){\makebox(0,0)[lb]{\raisebox{0pt}[0pt][0pt]{\twlrm o}}}
\put(250,587){\makebox(0,0)[lb]{\raisebox{0pt}[0pt][0pt]{\twlrm p}}}
\put(250,566){\makebox(0,0)[lb]{\raisebox{0pt}[0pt][0pt]{\twlrm q}}}
\put(250,545){\makebox(0,0)[lb]{\raisebox{0pt}[0pt][0pt]{\twlrm r}}}
\put(247,650){\makebox(0,0)[lb]{\raisebox{0pt}[0pt][0pt]{\twlrm m}}}
\put( 40,540){\framebox(60,60){}}
\put(195,670){\vector( 1, 0){ 45}}
\put( 90,570){\line( 1, 0){ 50}}
\put(140,570){\line( 0, 1){ 85}}
\put(140,655){\vector( 1, 0){ 45}}
\put(295,540){\framebox(25,140){}}
\put(250,670){\vector( 1, 0){ 45}}
\put( 90,550){\line( 0,-1){ 25}}
\put( 90,525){\line( 1, 0){255}}
\put(345,525){\line( 0, 1){130}}
\put(345,655){\vector(-1, 0){ 25}}
\put( 49,564){\makebox(0,0)[lb]{\raisebox{0pt}[0pt][0pt]{\twlrm head}}}
\put( 49,545){\makebox(0,0)[lb]{\raisebox{0pt}[0pt][0pt]{\twlrm tail}}}
\put(305,650){\makebox(0,0)[lb]{\raisebox{0pt}[0pt][0pt]{\twlrm s}}}
\put( 48,584){\makebox(0,0)[lb]{\raisebox{0pt}[0pt][0pt]{\twlrm 13}}}
\end{picture}
\vspace{0.4cm}
\begin{center}
{\large \bf Figure 23.4}
\end{center}
\vspace{0.2cm}


Since the last ``cblock'' ln Figure 23.3
was full, a new one has been obtained
from ``cfreelist'' and linked into the
list of ``cblock''s. The character count
and tail pointer have been adjusted appropriately.

\sbs{cinit (8234)}

This procedure, which is called once by
``main'' (1613), links the set of character buffers into the free list,
``cfreelist'', and counts the number of
character device types.

\bd
\item[8239:] ``ccp'' is the address of the first
word in the array ``cfree'' (8146);

\item[8240:] Round ``ccp'' up to the next
highest multiple of eight, and
mark out ``cblock'' sized pieces,
taking care not to exceed the
boundary of ``cfree''.

Note. In general there will be
``NCLIST -- 1'' (rather than
``NCLIST'') blocks so defined;

\item[8241:] Set the first word of the
``cblock'' to point to the current
head of the free list.

Note that ``c\_next'' is defined on
line 8141, and that the initial
value of ``cfreelist'' is ``NULL''.

\item[8242:] Update ``cfreelist'' to point to
the new head of the list;

\item[8244:] Count the number of character
device types. Upon reference to
``cdevsw'' on Sheet 46, it will be
seen that ``nchrdev'' will be set
to 16, whereas a more appropriate
value would be 10.
\ed

\sbs{getc (0930)}

This procedure is called by

\begin{verbatim}
   flushtty (8258, 8259, 8264)
   canon    (8292)    pcread  (8688)
   ttstart  (8520)    pcstart (8714)
   ttread   (8544)    lpstart (8971)
   pcclose  (8673)
\end{verbatim}

\noindent with a single argument which is the
address of a ``clist'' structure.

\bd
\item[0931:] Copy the parameter to r1 and save
the initial processor status word
and value of r2 on the stack;

\item[0934:] Set the processor priority to
five (higher than the interrupt
priority of a character device);

\item[0936:] r1 points to the first word of a
``clist'' structure (i.e. a character count). Move the second word
of this structure (i.e. a pointer
to the head character) to r2;

\item[0937:] If the list is empty (head
pointer is ``NULL'') go to line 0961;

\item[0938:] Move the head character to r0 and
increment r2 as a side effect;

\item[0939:] Mask r0 to get rid of any
extended negative sign;

\item[0940:] Store the updated head pointer
back in the ``clist'' structure.
(This may have to be altered
later.);

\item[0941:] Decrement the character count and
if this is still positive, go to
line 0947;

\item[0942:] The list is now empty, so reset the head and tail
character pointers to ``NULL''. Go to line 0952;

\item[0947:] Look at the three least significant bits of r2. If these are
non-zero, branch to line 0957
(and return to the calling routine forthwith);

\item[0949:] At this point, r2 is pointing at
the next character position
beyond the ``cblock''. Move the
value stored in the first word of
the ``cblock'' (i.e. at r2 -- 8),
which is the address of the next
``cblock'' in the list to the head
pointer in the ``clist''. (Note
that r1 was incremented as a side
effect at line 0941):

\item[0950:] The last value stored needs to
incremented by two (Consult
Figures 23.2 and 23.3);

\item[0952:] At this point, a
``cblock'' determined by r2 is to be returned to
``cfreelist''. Either r2 points
into the ``cblock'' or just beyond
it. Decrement r2 so that r2 will
point into the ``cblock'';

\item[0953:] Reset the three least significant
bits of r2, leaving a pointer to
the ``cblock'';

\item[0954:] Link the ``cblock'' into ``cfreelist'';

\item[0957:] Restore the values of r2 and PS
from the stack and return;

\item[0961:] At this point the list is known
to be empty because a ``NULL'' head
pointer was encountered. Make
sure that the tail pointer is
``NULL'' also;

\item[0962:] Move --1 to r0 as the result to be
returned when the list is empty.
\ed

\sbs{putc (0967)}

This procedure is called by

\begin{verbatim}
   canon     (8323)
   ttyinput  (8355, 8358)
   ttyoutput (8414, 8478)
   pcrint    (8730)
   pcoutput  (8756)
   lpoutput  (8990)
\end{verbatim}

\noindent with two arguments: a character and the
address of a ``clist'' structure.

``getc'' and ``putc'' have related functions and
the codes for the two procedures are similar in many respects.
For this reason the code for ``putc''
will not be examined in detail, but is
left for the reader.

It should be noted that ``putc'' can fail
if a new ``cblock'' is needed and
``cfreelist'' is empty. In this case a
non-zero value (line 1002) is returned
rather than a zero value (line 0996).

{\bf Note.} The procedures ``getc'' and ``putc''
discussed here are {\bf NOT} directly related
to the procedures d1scussed in the Sections
``GETC(III)'' and ``PUTC(III)'' of the UPM.

\sbs{Character Sets}

UNIX makes use of the full ASCII character set,
which is displayed in Section ``ASCII(V)'' of the UPM. Since
knowledge of this character set is
often assumed without comment, not
always justifiably, some comment here
would seem to be in order.

``ASCII'' is an acronym for ``American
Standard Code for Information Interchange''.

\sbs{Control Characters}

The first 32 of the 128 ASCII characters are non-graphic and are intended
for the control of some aspect of
transmission or display. The control
characters explicitly used or recognised by UNIX are

\noindent\begin{tabular}{llll}\\
{\bf Numeric} & & \multicolumn{1}{c}{\bf Description} & {\bf UNIX} \\
{\bf Value} &	&	& {\bf Name} \\ \hline
004 & eot & end of transmission		& 004 \\
    &    & or (control-D) & \\
010 & bs & back space			& 010 \\
011 & ht & (horizontal) tab		& '\verb+\+t' \\
012 & nl & new line or line feed	& FORM \\
014 & np & new page or form feed	& '\verb+\+r' \\
015 & cr & carriage return		& '\verb+\+n' \\
034 & fs & file separator or quit	& CQUIT \\
040 & sp & forward space or blank	& '~' \\
0177 & del & delete			& CINTR
\end{tabular}


It will be noted that the last two of
these belong to the last 96 characters
or the graphic portion, of the code.

\sbs{Graphic Characters}

There are 96 graphic characters. Two of
these, the space and the delete, are
not ``visible'', and may be ciassified
with the control characters.

The graphic characters may be divided
into three groups of 32 characters,
which may be roughly characterised as

\bi
\item numeric and special characters
\item upper case alphabetic characters
\item lower case alphabetic characters.
\ei

Of course, since there are only 26
alphabetic characters, the latter two
groups include some special characters
as well. In particular, the last group
includes the following six nonalphabetic characters:

\begin{tabular}{lll}\\
140 & `			& reverse apostrophe \\
173 & \{		& left brace \\
174 & \verb+|+		& vertical bar \\
175 & \}		& right brace \\
176 & \verb+~+		& tilde \\
177 &			& delete \\
\end{tabular}

\sbs{Graphic Character Sets}

Devices such as line printers or terminals which support {\bf all} the ASCII
graphic symbols are often said to support the 96 ASCII character set (though
there are only 94 graphics actually
involved).

Devices which support all the ASCII
graphic symbols except those in the
last group of 32, are said to support
the 64 ASCII character set. Such devices lack the lower case alphabetics
and the symbols listed above, namely ``\verb+~+'', ``\{'', ``\verb+|+'' and
``\verb+\}+''. Note that
``delete'', since it is not a visible
character, can still be supported.

Devices in this latter group may be
referred to as ``upper case only''.

Sometimes some of the graphic symbols
may be non-standard, e.g. $\leftarrow$  instead
of \_ and this can be inconvenient,
though not usually fatal.

UNIX prefers, as the reader is no doubt
well aware, to view the world through
``lower case'' spectacles. Alphabetic
characters received from an ``upper case
only'' terminal are translated
immediately upon receipt from upper
case to lower case. A lower case alphabetic may subsequently be translated
back to upper case if it is preceded by
a single backslash. For output to such
a terminal, both upper and lower case
alphabetic characters are mapped to uppercase.

%Equivalences for the five ``upper case''
%special characters are as follows:

The conventions for line printers and
terminals are different because:

\bd
\item[(a)] for line printers, horizontal
alignment is usually important,
and it is possible (without too
much difficulty) to print composite, overstruck characters
(using the minus sign in this
case); and

\item[(b)] for terminals, horizontal alignment is not considered to be so
important; backspacing to provide overstruck characters does
not work on most VDUs; and,
since the same graphic conventions are used for both input
and output, the symbols should
be as convenient to type as possible.
\ed

\sbs{maptab (8117)}

This array is used in the translation
of character input from a terminal preceded by a single backslash, ``\verb+\+''.

There are three characters, 004 (eot),
`\#' and `@', which always have special
meanings and need to be asserted by a
backslash whenever they are to be
interpreted literally. These three
characters occur in ``maptab'' in their
``natural'' locations (i.e. their locations in the ASCII table). Thus for
example `\#' has code 043 and

\begin{verbatim}
   maptab[043] == 043.
\end{verbatim}

The other non-null characters in ``maptab'' are involved in the translation of
input characters from ``upper case only''
devices and do not occur in their
``natural'' locations but in the location
of their equivalent character, e.g. ``\verb+\+\{''
occurs in the natural location for ``\{'',
since ``\verb+\+\{'' will be interpreted as ``\{'',
etc.

Note the situation regarding alphabetic
characters. This is only explicable
when it is remembered that the alphabetic characters are all translated to
lower case before any backslash is
recognised.

\sbs{partab (7947)}

This array consists of 256 characters,
like ``maptab''. Unfortunately the initialisation of ``partab''
was omitted from the UNIX Operating System Source Code
booklet. It is certainly needed, and so
is given now:

\begin{verbatim}
  char partab [] {

0001,0201,0201,0001,0201,0001,0001,0201,
0202,0004,0003,0205,0005,0206,0201,0001,
0201,0001,0001,0201,0001,0201,0201,0001,
0001,0201,0201,0001,0201,0001,0001,0201,
0200 0000,0000,0200,0000,0200,0200,0000,
0000 0200,0200 0000,0200,0000,0000,0200,
0000,0200,0200 0000,0200,0000,0000,0200,
0200,0000,0000,0200,0000,0200,0200,0000,
0200,0000,0000,0200,0000,0200,0200,0000,
0000,0200,0200,0000,0200,0000,0000,0200,
0000,0200,0200,0000,0200,0000,0000,0200,
0200 0000,0000 0200,0000,0200,0200,0000,
0000 0200,0200 0000,0200,0000,0000,0200,
0200,0000,0000,0200,0000,0200,0200,0000,
0200,0000,0000,0200,0000,0200,0200,0000,
0000,0200,0200,0000,0200,0000,0000,0201
   };
\end{verbatim}

Each element of ``partab'' is an eight
bit character, which, with the use of
appropriate bitmasks (0200 and 0177),
can be interpreted as a two part structure:

\begin{tabular}{ll}\\
bit 7    & parity bit;\\
bits 3-5 & not used. Always zero;\\
bits 0-2 & code number.\\
\end{tabular}

\bigskip

The parity bit is appended to the seven
bit ASCII code when a character is
transmitted by the computer, to form an
eight bit code with even parity.

The code number is used by ``ttyoutput''
(8426) to classify the character into
one of seven categories for determining
the delay which should ensue before the
transmission of the next character.
(This is particularly important for
mechanical printers which require time
for the carriage to return from the end
of a line, etc.)
